\documentclass[a4paper,12pt]{report}
\usepackage[utf8]{inputenc}
\usepackage[T1]{fontenc}
\usepackage[italian]{babel}
\usepackage{amsmath}
\usepackage{listings}
\usepackage{amssymb}
\usepackage{cancel}
\usepackage{tikz}
\usetikzlibrary{angles, quotes}
\usepackage{tikz-3dplot}
\usepackage{mathptmx}
\usepackage{tzplot}
\newcommand{\taninv}{\tan^{-1}}
\usepackage{titlesec}
\usepackage{multicol}
\usepackage[only,llbracket,rrbracket]{stmaryrd}
\newcommand\val[1]{\llbracket#1\rrbracket}
\newcommand\Iff{\Leftrightarrow}

\newcommand\qed{\begin{flushright}{$\square$}\end{flushright}}


\titleformat{\chapter}
	{\Large\bfseries}		% format
	{}							% label
	{0pt}						% sep
	{\huge}					% before-code

\begin{document}
\title{Appunti del corso di Logica 

A.A. 2023/2024}
\author{Note a cura di Niccol\`{o} Iselle\\
Corso della Prof. Alessandra Di Pierro}
\date{}
\maketitle
\tableofcontents

\chapter{Introduzione}
% ============================= LEZIONE 1 - 2/10/23 =============================
\section{Logica - `Studio del ragionamento'}
Possiamo suddividere la \emph{logica} in due categorie: quella \emph{formale}, ovvero \emph{simbolica}, che \`{e} lo sudio dei passaggi del nostro ragionamento basati su connettivi nelle nostre `sentenze' (\emph{if, and, or}); e quella \emph{informale} (non simbolica), che \`{e} lo studio del pensiero logico in un contesto informale come la critica o l'argomentazione, pi\`{u} inerente al campo filosofico. 

\section{`La barretta di cioccolata'}
Consideriamo una barretta di cioccolata formata da 24 quadratini disposti in un
rettangolo 6 x 4. Il nostro obbiettivo \`{e} quello di separare ogni quadratino
col minor numero di tagli (sempre effettuati lungo le linee). Quanti tagli serviranno?

\paragraph{Soluzione:} intuitivamente serviranno tanti tagli quanti sono i quadratini di cioccolata meno uno.

\subsection{Dimostrazione per induzione}
\paragraph{Principio di induzione:} 
\begin{itemize}
\item $ P(n) $
\item $ P(0) $
\item $ P(n-1) $ ipotesi induttiva, se assumendo che $P$ \`{e} vera su $n-1$.
\item Dimostriamo che $P(n)$, allora $P$ vale per ogni $n$.
\end{itemize}
\paragraph{Dimostrazione:}

\begin{enumerate}
\item Se la barretta \`{e} fatta da un quadratino $\rightarrow$ banale, 0 tagli.
\item Assumendo che per una barretta composta da $1 < m < N$ quadratini abbiamo gi\`{a} dimostrato che servono esattamente $m - 1$ tagli per una barretta composta da $m$ quadratini. 
\item Se ora abbiamo una barretta da $N$ quadratini, e la dividiamo in due parti $m_1$ ed $m_2$. Ovviamente $m_1 + m_2 = N$. Per l'ipotesi induttiva, ci serviranno $m_1 - 1$ tagli per separare i quadratini di $m_1$ ed $m_2 - 1$ tagli per separare i quadratini di $m_2$. Il totale sar\`{a} quindi \[1 + (m_1 - 1) + (m_2 -1) = N - 1\]
\end{enumerate}

\subsection{Dimostrazione per invariante}
\paragraph{Invariante:} si tratta di una propriet\`{a} che non cambia il suo valore, ovvero una proposizione \emph{vera}. Ad esempio, in un programma l'invariante \`{e} quella propriet\`{a} che rimane uguale prima, durante e dopo l'esecuzione del programma stesso.
\newline

Per dimostrare che servono $N-1$ tagli per separare la barretta di cioccolata nei singoli quadratini. Possiamo notare che ogni volta che rompiamo la barretta il numero totale di pezzi aumenta di uno (il pezzo pi\`{u} grande viene diviso in due pezzi pi\`{u} piccoli). Quando non abbiamo pi\`{u} pezzi da rompere, ogni pezzo \`{e} un quadratino. All'inizio, dopo 0 tagli, abbiamo 1 pezzo. Dopo aver fatto 1 taglio, otteniamo 2 pezzi. Aumentare il numero dei tagli di 1 fa aumentare il numero di pezzi di 1. Quindi il secondo numero (numero di quadratini) sar\`{a} sempre di 1 pi\`{u} grande del primo (numero di tagli).

\section{Concetto di insieme}

Possiamo definire un insieme in modo estensivo
\[ A = \{a, b, c\} \]
elencando tutti i suoi elementi, oppure possiamo definirlo utilizzando una propriet\`{a} condivisa da tutti i suoi elementi
\[ A = \{x | x \text{ \`{e} pari}\} \subseteq \mathbb{N} \]
che equivale a scrivere
\[ \{x | x \in \mathbb{N} \text{ e $x$ \`{e} pari}\} \]
o ancora
\[ \{x \in \mathbb{N} | x \text{ \`{e} pari}\} \]

\section{Logica proposizionale (o sentenziale)}

La logica proposizionale \`{e} un ramo della logica simbolica deduttiva che assume le proposizioni (sentenze) come unit\`{a} fondamentali dell'analisi logica. Le preposizioni che prendiamo in esame sono le \emph{asserzioni}.

\paragraph{Asserzione:} \`{e} una proposizione che ha un valore di verit\`{a}, cio\`{e} pu\`{o} assumere il valore \emph{vero (T)} oppure il valore \emph{falso (F)}.

\paragraph{Esempi di asserzioni:}
\begin{itemize}
\item Verona \`{e} una citt\`{a} del Veneto
\item Fuori sta piovendo
\item Nel 5024 il Sole si spegner\`{a}
\end{itemize}

Le asserzioni possono essere collegate da \emph{connettivi logici} tra loro. Prendiamo in esempio la frase `\emph{Se \underline{c'\`{e} il sole} allora \underline{serve la protezione}. Ma se \underline{serve la protezione} allora
\underline{non si pu\`{o} fare il bagno}.}'
Possiamo estrapolare tre asserzioni:
\begin{itemize}
\item S: \emph{C'\`{e} il sole}
\item P: \emph{Serve la protezione}
\item B: \emph{Non si pu\`{o} fare il bagno}
\end{itemize}
e possiamo collegarle tra loro con i connettivi logici, per formare lo stesso significato espresso dalla frase, nel modo in cui segue:
\[ S \rightarrow P \]
\[ P \rightarrow B \]
(Dove $\rightarrow$ si legge \emph{implica}).

\subsection{Linguaggi formali - (automi di riconoscimento)}

Nella logica dobbiamo usare un linguaggio formale, formato da un \emph{alfabeto}. Un esempio di definizione di alfabeto potrebbe essere il seguente:
\[ A = \{a, b\} \]
Un linguaggio formale, basato sull'alfabeto $A$ potrebbe essere:
\[ L(A) = \{a, b, ab, aa, ba, \dots \} \]
Notare che $ab \ne ba$, si dice che sono \emph{parole} diverse.
Un'altra notazione importante \`{e} \emph{A-star} scritta come $A^{*}$, che sta ad indicare tutte le stringhe che si possono costruire con l'alfabeto A.
\paragraph{Esempio:}

Dato l'alfabeto $A = \{a, b\}$, costruiamo un linguaggio costituito da tutte le stringhe formate da un numero pari di simboli:
\[ L_p = \{aa, ab, abab, \dots\} \]
Possiamo quindi affermare che $a \notin L_p$.


\section{Teoria dell'Aritmetica di Peano - Definizione dei numeri naturali}

Per definire i numeri naturali, utilizziamo una \emph{struttura} formata da tre elementi:
\[ <\mathbb{N}, 0, succ> \]
\begin{itemize}
\item $\mathbb{N}$ \`{e} l'universo del discorso
\item $0$ \`{e} una costante, il caso base
\item $succ$ \`{e} una funzione su $\mathbb{N}$ definita come \[ succ: \mathbb{N} \rightarrow \mathbb{N} \]
\end{itemize}

\subsection{I ASSIOMA}

\begin{center} Esiste un numero $0 \in \mathbb{N}$ \end{center}
Il primo assioma dice che $0$ \`{e} un elemento privilegiato di $\mathbb{N}$ detto \emph{zero}.

\subsection{II ASSIOMA}
\begin{center} Esiste una funzione $succ:  \mathbb{N} \rightarrow \mathbb{N}$ \end{center}

Il secondo assioma dice che $succ$ \`{e} un'operazione unaria iniettiva su A;

\subsection{III ASSIOMA}
\[ succ(x) \ne 0 \text{ per ogni } x \in \mathbb{N} \]
Il terzo assioma ci dice che 0 non \`{e} il successore di nessun numero, un altro modo di esprimerlo \`{e} $0 \notin Im(succ)$.

\subsection{IV ASSIOMA}
\begin{center} Se $P \subseteq \mathbb{N}$ e valgono le seguenti propriet\`{a}:
\begin{itemize}
\item $0 \in P$
\item $\forall n \in \mathbb{N}.(n \in P \rightarrow succ(n) \in P)$,
\end{itemize}
allora $P = \mathbb{N}$.
\end{center}
Il quarto assioma sta ad indicare che, partendo dal fatto che 0 \`{e} un numero naturale, se $n$ \`{e} un numero naturale allora anche $n+1$ \`{e} un numero naturale. Questo assioma \`{e} detto \emph{assioma di induzione}.

\section{Principio di Induzione}
Dato che una propriet\`{a} $P$ sui numeri naturali non \`{e} altro che un sottoinsieme di $\mathbb{N}$, ovvero $P\subseteq\mathbb{N}$, possiamo riformulare l'assioma di induzione come principio per provare propriet\`{a} sui naturali.
\subsection{Principio di induzione:}
Sia $P(x)$ una propriet\`{a} su $\mathbb{N}$. \newline
Se $P(0)$ e $\forall n \in \mathbb{N}.(P(n) \rightarrow P(succ(n))$ allora $\forall m \in \mathbb{N}.P(m)$.

\subsection{Esempio}
Dimostrare per induzione su $n$ che 
\[ P(n) \equiv \sum_{i=0}^{n}2i+1 = (n+1)^2 \]

\paragraph{Caso base $n=0$:}
\[ \sum_{i=0}^{0} 2i+1 = (n+1)^2 \]
\[ 2*0+1 = (0 + 1)^2 = 1 \]

\paragraph{Ipotesi induttiva:} supponiamo ora che valga 
\[ \sum_{i=0}^{n-1} 2i+1 = ((n-1)+1)^2 \]
e utilizziamo questa \emph{ipotesi induttiva} per dimostrare che vale anche per $n$.

\paragraph{Passo induttivo:}

Partiamo dalla formula che dobbiamo dimostrare 
\[ \sum_{i=0}^{n}2i+1 = (n+1)^2 \]
e possiamo riscriverla come
\[ \sum_{i=0}^{n-1} 2i+1 + 2n+1 = (n+1)^2 \]
ora possiamo utilizzare l'ipotesi induttiva, sapendo che 
\[ ((n-1)+1)^2 + 2n+1 = n^2 + 2n + 1 \]
e risolviamo:
\[ (n-1)^2 + 2(n-1) + 1 + 2n + 1 = n^2+2n+1 \] 
\[ n^2-2n+1+2n-2+1+2n+1 = n^2+2n+1 \]
\[ n^2+2n+1 = n^2+2n+1 \]
\begin{flushright}{$\square$}\end{flushright}


% ============================= LEZIONE 2 - 5/10/23 =============================
\section{Cardinalit\`{a} di un insieme}
La cardinalit\`{a} di un insieme \`{e} il numero di elementi che esso contiene, per esempio:
\[ \mathbb{N} = \{ 1, 2, 3, \dots \} \]
La cardinalit\`{a} di $\mathbb{N}$, che si esprime in notazione come $|\mathbb{N}|$ \`{e} un \emph{infinito numerabile}.
Nel caso di un insieme con un numero finito di elementi, ad esempio $A = \{a, b, c \}$ la sua cardinalit\`{a} $|A| = 3$.

Se un insieme ha la stessa cardinalit\`{a} di $\mathbb{N}$ significa che pu\`{o} essere messo in \emph{corrispondenza biunivoca} con $\mathbb{N}$.
Ovvero 
\[ \forall b\in B \exists n \in \mathbb{N} \]
\[ f: b \rightarrow n \]
E vale anche che
\[ \forall b \in \mathbb{N} \exists b \in B \]
\[ f: n \rightarrow b \]

\section{Numeri naturali}
Come precedentemente detto, i numeri naturali sono formati da
\[ \left<\mathbb{N}, succ, 0\right> \]
$succ: \mathbb{N} \rightarrow \mathbb{N}$ \`{e} una funzione \emph{iniettiva}, ovvero ad ogni elemento del dominio corrisponde uno e uno solo elemento dell'immagine. 
Presi due insiemi $A$ e $B$, se $Im(f) = B$ si dice che la funzione \`{e} \emph{surgettiva}. 
Se una funzione \`{e} \emph{surgettiva} e \emph{iniettiva} si ha una \emph{corrispondenza biunivoca (biezione)}.
\subsubsection{Esempio di proposizione sui numeri naturali}
\[P:= n \text{ \`{e} pari} \]
\[ P = {n \in \mathbb{N} | P(n) } \]
\[ = {n \in \mathbb{N} | n \text{ \`{e} pari}} \] 

\subsection{Definizioni induttive}

\subsubsection{Esempio}
Definiamo $h$ come segue
\[ h: \mathbb{N}x\mathbb{N} \rightarrow \mathbb{N}\thickspace \text{ t.c. }  h(x,y) = succ(x) * y \]

\paragraph{caso base} \[f(0) = 1 \]
\[f(succ(n) = h(n, f(n)) = succ(n) * f(n) \]

\paragraph{n = 0}
\[ succ(0) * f(0) = 1 * 1 = 1 \]

\paragraph{caso induttivo}
\[ f(succ(succ(0))) = succ(n) * f(n) \]

\paragraph{n = 2}
\[ f(succ(succ(0)) = f(2) = 2 * 1 = 2 \]
\paragraph{n = 3}
\[ f(succ(succ(succ(0)))) = f(3) = 3 * 2 * 1 \]
\[ \dots \]
Possiamo notare che questa \`{e} la definizione induttiva del fattoriale di un numero.
\chapter{Sintassi}
\section{Simboli e significato}
\begin{tabular}{| c | c |}
\hline
\textbf{Simboli} & \textbf{Significato} \\ 
\hline
$P \to Q$ & \emph{se} P \emph{allora} Q \\
\hline
$P \leftrightarrow Q$ & P \emph{se e solo se} (\emph{sse}) Q \\
\hline
$P \wedge Q$ & P \emph{and}/\emph{e} Q \\
\hline
$P \lor Q$ & P \emph{or}/\emph{oppure} Q \\
\hline
$\neg P$ & \emph{not} P (P \`{e} \emph{falso}) ($(\neg P) \equiv (P \to \bot)$) \\
\hline
$\forall x P$ & \emph{per ogni elemento x P(x) \`{e} vero} \\
\hline
$\exists x P$ & \emph{Esiste un elemento tale che P \`{e} vero per quell'elemento} \\
\hline
\end{tabular}

% ============================= LEZIONE 3 - 9/10/23 =============================
\section{Linguaggio formale}
Il linguaggio della logica proposizionale ha un alfabeto che consiste di 
\begin{enumerate} 
\item proposizioni (o simboli proposizionali): $p_0, p_1, p_2, \dots$
\item connettivi: $\wedge, \lor, \to, \leftrightarrow, \neg, \bot, \top$
\item simboli ausiliari: `(` e `)'
\end{enumerate}

\subsubsection{Bottom e Top}
\begin{itemize}
\item Bottom \`{e} il falso e si indica con $\bot$, il suo valore \`{e} sempre 0.
\item Top \`{e} il vero e si indica con $\top$, il suo valore \`{e} sempre 1 e viene definito come $\vDash \top \leftrightarrow (\bot \to \bot)$.
\end{itemize}

\subsection{Proposizioni atomiche e proposizioni composte}
\subsubsection{Proposizioni atomiche o minimali}
Le proposizioni atomiche sono quelle proposizioni che non possono essere suddivise in parti pi\`{u} piccole. Per esempio \[ 5 \in \{0, 1, 2, 3, 4, 5\} \]
\`{e} una proposizione atomica.

\subsubsection{Proposizioni composte} 
\[ c \text{ \`{e} razionale oppure $c$ \`{e} irrazionale} \]
Una proposizione si dice composta quando pu\`{o} essere scomposta in sotto-proposizioni pi\`{u} piccole, unite da connettivi logici ($\wedge, \lor, \to, \dots $).

\subsection{L'insieme $PROP$ delle proposizioni}
L'insieme $PROP$ \`{e} il \emph{pi\`{u} piccolo insieme} $X$ con le seguenti propriet\`{a}:
\begin{enumerate}
\item $p_i \in X$ per $i \in \mathbb{N}$ e $\bot \in X$
\item $\varphi, \psi \in X \to (\varphi \wedge \psi), (\varphi \lor \psi), (\varphi \to \psi) , (\varphi \leftrightarrow \psi) \in X$
\item $\varphi \in X \to (\neg\varphi) \in X$
\end{enumerate}
Il connettivo \emph{bottom} ($\bot$) \`{e} una costante logica, con valore sempre falso.

\subsection{Priorit\`{a} dei connettivi}
Le formule devono avere le parentesi, ma per poter scrivere in maniera abbreviata una formula senza le parentesi, e per poterla leggere correttamente, dobbiamo associare delle priorit\`{a} ai connettivi.
\begin{enumerate}
\item $\neg$ \`{e} il connettivo con pi\`{u} priorit\`{a}
\item $\wedge, \lor$
\item $\to, \leftrightarrow$
\end{enumerate}

\paragraph{Esempi:}
\begin{itemize}
\item $\neg\varphi \lor \varphi$ \`{e} l'abbreviazione della formula $((\neg\varphi)\lor \varphi)$
\item $\neg(\neg\neg\neg\varphi\wedge\bot)$ \`{e} l'abbreviazione della formula $(\neg((\neg(\neg(\neg\varphi))) \wedge \bot))$
\item $\varphi \lor \psi \to \varphi$ significa $((\varphi \lor \psi)\to \varphi)$
\end{itemize}
% ============================ LEZIONE 4 - 12/10/23 ============================
\section{Parse Tree}
Nella definizione ricorsiva di formule si pu\`{o} usare la rappresentazione ad albero:
\[T(\varphi) = \varphi \text{ se } \varphi \in AT \]
Questo \`{e} l'albero formato dal solo nodo $\varphi$:
\begin{center}
\begin{tikzpicture}
\filldraw[black] (0,0) circle (2pt);
\node at (0,0) [right] {$\varphi$};
\end{tikzpicture}
\end{center}
Abbiamo quindi che $T(\neg\varphi)$ \`{e} uguale a:
\begin{center}
\begin{tikzpicture}
\filldraw[black] (0,0) circle (2pt);
\node at (0,0) [right] {$(\neg \varphi)$};
\draw (0, -0.1) -- (0, -0.9);
\filldraw[black] (0,-1) circle (2pt);
\node at (0,-1) [right] {$T(\varphi)$};
\node at (1.5, -1) [right] {$\leftarrow$ \emph{sottoalbero di} $\varphi$};
\end{tikzpicture}
\end{center} 
Analogamente per $T(\varphi \square \psi)$, con $\square \in \{\wedge, \lor, \to\}$ l'albero corrispondente sar\`{a}:
\begin{center}
\begin{tikzpicture}
\filldraw[black] (0,0) circle (2pt);
\node at (0.2,0) [right] {$\varphi \square \psi$};
\draw (-0.1, -0.1) -- (-1, -0.9);
\filldraw[black] (-1, -1) circle (2pt);
\node at (-1, -1) [left] {$T(\varphi)$};
\draw (0.1, -0.1) -- (1, -0.9);
\filldraw[black] (1, -1) circle (2pt);
\node at (1, -1) [right] {$T(\psi)$};
\node at (-2, -1) [left] {\emph{sottoalbero di} $\varphi \to$};
\node at (2, -1) [right] {$\leftarrow$ \emph{sottoalbero di} $\psi$};
\end{tikzpicture}
\end{center}

\paragraph{Esercizio:} costruire l'albero della formula $\varphi = (p_1 \to (\bot \lor (\neg p_3)) $. 
\begin{center}
\begin{tikzpicture}
%phi - radice
\filldraw[black] (0,0) circle (2pt);
\node at (0.2,0) [right] {$\varphi$};
% p1, ramo sinistro
\draw (-0.1, -0.1) -- (-1.9, -0.9);
\filldraw[black] (-2,-1) circle (2pt);
\node at (-2, -1) [left] {$p_1$};
% bottom or not p3, ramo destro
\draw (0.1, -0.1) -- (1.9, -0.9);
\filldraw[black] (2, -1) circle (2pt);
\node at (2.2, -1) [right] {$(\bot \lor (\neg P_3))$};
% bottom, foglia del ramo destro
\draw (1.9, -1.1) -- (0.1, -1.9);
\filldraw[black] (0, -2) circle (2pt);
\node at (0, -2) [left] {$(\bot)$};
% not p3, sottoramo destro
\draw (2.1, -1.1) -- (3.9, -1.9);
\filldraw[black] (4, -2) circle (2pt);
\node at (4, -2) [right] {$(\neg p_3)$};
%foglia p_3
\draw (4.1, -2.1) -- (4.1, -3.9);
\filldraw[black] (4, -4) circle (2pt);
\node at (4, -4) [right] {$p_3$};
\end{tikzpicture}
\end{center}
Al posto delle formule possiamo usare, in modo totalmente equivalente, i connettivi:
\begin{center}
\begin{tikzpicture}
%phi - radice
\filldraw[black] (0,0) circle (2pt);
\node at (0.2,0) [right] {$\to$};
% p1, ramo sinistro
\draw (-0.1, -0.1) -- (-1.9, -0.9);
\filldraw[black] (-2,-1) circle (2pt);
\node at (-2, -1) [left] {$p_1$};
% bottom or not p3, ramo destro
\draw (0.1, -0.1) -- (1.9, -0.9);
\filldraw[black] (2, -1) circle (2pt);
\node at (2.2, -1) [right] {$\lor$};
% bottom, foglia del ramo destro
\draw (1.9, -1.1) -- (0.1, -1.9);
\filldraw[black] (0, -2) circle (2pt);
\node at (0, -2) [left] {$(\bot)$};
% not p3, sottoramo destro
\draw (2.1, -1.1) -- (3.9, -1.9);
\filldraw[black] (4, -2) circle (2pt);
\node at (4, -2) [right] {$\neg$};
%foglia p_3
\draw (4.1, -2.1) -- (4.1, -3.9);
\filldraw[black] (4, -4) circle (2pt);
\node at (4, -4) [right] {$p_3$};
\end{tikzpicture}
\end{center}

Possiamo definire un albero come un sottoinsieme finito dell'insieme di tutte le sequenze ($Seq$) che possiede le seguenti propriet\`{a}:
\begin{itemize}
\item $T \subset Seq$
\item Se $t \in T$ e $s \le t$, allora $s \in T$
\item $u = s * t$ significa che se 
\[ s = (n_1, n_2, \dots, n_i) \text{ e } t = (m_1, m_2, \dots, m_j) \text{ allora}\]
\[ u = (n_1, n_2, \dots, n_i, m_{i+1}, \dots, m_j) \]
\end{itemize}

\paragraph{Nota:} Nella deduzione naturale gli alberi indicano derivazioni logiche e sono alebri rovesciati. 

\chapter{Semantica}
\section{Valutazione}
Si definisce valutazione atomica $v$, una funzione tale che:
\begin{itemize}
\item $v: AT \to \{0, 1\}$
\item $v(\bot) = 0$
\end{itemize}
Le vlutazioni sono infinite (infinito numerabile).

\subsection{Tabelle di verit\`{a}}
\begin{multicols}{3}
\subsubsection{Or - $p_1\lor p_2$} 
\begin{tabular}{c c | c}
$p_1$ & $p_2$ & $p_1 \lor p_2$ \\
\hline
$0$ & $0$ & $0$ \\
$0$ & $1$ & $1$ \\
$1$ & $0$ & $1$ \\
$1$ & $1$ & $1$ 
\end{tabular}

\subsubsection{And - $p_1\wedge p_2$} 
\begin{tabular}{c c | c}
$p_1$ & $p_2$ & $p_1 \wedge p_2$ \\
\hline
$0$ & $0$ & $0$ \\
$0$ & $1$ & $0$ \\
$1$ & $0$ & $0$ \\
$1$ & $1$ & $1$ 
\end{tabular}
\subsubsection{Implicazione - $p_1\to p_2$} 
\begin{tabular}{c c | c}
$p_1$ & $p_2$ & $p_1 \to p_2$ \\
\hline
$0$ & $0$ & $1$ \\
$0$ & $1$ & $1$ \\
$1$ & $0$ & $0$ \\
$1$ & $1$ & $1$ 
\end{tabular}
\end{multicols}
\begin{multicols}{2}
[
\subsubsection{Not - $\neg p$}
La negazione ($\neg$) \`{e} intesa come abbreviazione per $p \to \bot$, possiamo quindi ricavarne due tabelle di verit\`{a}:
]
\subsubsection{$p \to \bot$}
\begin{tabular}{c c | c}
$p$ & $\bot$ & $p \to \bot$ \\
\hline
$0$ & $0$ & $1$ \\
$1$ & $0$ & $0$ 
\end{tabular}
\subsubsection{$\neg p$}
\begin{tabular}{c | c}
$p$ & $\neg p$ \\
\hline
$0$ & $1$ \\
$1$ & $0$ 
\end{tabular}
\end{multicols}

\section{Valutazione su $PROP$}
Dopo aver definito la valutazione per le formule atomiche, possiamo definire la valutazione per formule logiche composte.
\subsubsection{Definizione}
\[ \val{ \cdot }_v: PROP \to \{0, 1\} \]
$\val{\varphi}_v$ \`{e} una valutazione in $PROP$ se:
\begin{itemize}
\item $\val{\varphi \wedge \psi}_v = 1 \Iff \val{\varphi}_v = 1 \text{ AND } \val{\psi}_v = 1$
\item $\val{\varphi \lor \psi}_v = 1 \Iff \val{\varphi}_v = 1 \text{ OR } \val{\psi}_v = 1$
\item $\val{\varphi \to \psi}_v = 1 \Iff \val{\varphi}_v = 0 \text{ OR } \val{\psi}_v = 1$ \\
$\val{\varphi \to \psi}_v = 0 \Iff \val{\varphi}_v = 1 \text{ AND } \val{\psi}_v = 0$
\item $\val{\neg\varphi}_v = 1 \Iff \val{\varphi}_v = 0$
\end{itemize}
\paragraph{Nota:} due formule sono \emph{equivalenti} se hanno \emph{lo stesso valore di verit\`{a}}.

% ============================ LEZIONE 5 - 16/10/23 ============================ 
\subsection{Proposizione}
Per ogni valutazione atomica $v$, esiste un'unica semantica $\val{\cdot}_v : PROP \to \{0, 1\}$ tale che $\val{\varphi}_v = v(\varphi)$.

\section{Conseguenza logica}
Sia $\Gamma$ (\emph{gamma}) un'insieme di formule proposizionali, ovvero $\Gamma \subseteq PROP$, e sia $\varphi \in PROP$ una formula. La dicitura \emph{$\varphi$ \`{e} conseguenza logica di $\Gamma$} si denota con
\[ \Gamma \vDash \varphi \]
e indica che da $\Gamma$ segue logicamente $\varphi$, ovvero:
\[ \forall v, \thickspace \val{\Gamma}_v= 1 \implies \val{\varphi}_v = 1 \]
A parole, significa che se la valutazione di tutte le formule nell'insieme $\Gamma$ \`{e} vera, allora anche $\varphi$ \`{e} vera.
\subsection{Esempo di conseguenza logica}
\[ \{\varphi, \psi\} \vDash \varphi \wedge \psi \]
\[ \forall v,\text{ se } \thickspace \footnote[1]{Definizione di and}\val{\varphi}_v = \val{\psi}_v = 1, \thickspace \text{ allora} 
\]
\[ \val{\varphi \wedge \psi}_v = 1 \]
\qed

 La dicitura $\val{\Gamma}_v = 0$ invece significa che $\exists \gamma \in \Gamma : \val{\gamma}_v = 0 $
ovvero, basta che ci sia una formula in $\Gamma$ che per la valutazione $v$ non sia vera, per affermare che $\val{\Gamma}_v = 0$. 
\paragraph{Esempio:} $\Gamma = \{p, \neg p\}$, $p \in AT$. Non possiamo affermare che $\val{\Gamma}_v = 1$ perch\`{e} $\val{p}_v = 1 \implies \val{\neg p}_v = 0$, e viceversa.

\section{Tautologia}
Una \emph{tautologia} \`{e} una conseguenza logica senza premesse, quindi per ogni valutazione $v$, dobbiamo avere che $\val{\varphi}_v = 1$. Si indica con $\vDash \varphi$.
\subsection{Esempi di tautologia}
\begin{itemize}
\item $\vDash \varphi \lor \neg \varphi$
\[ \forall v \thickspace \val{\varphi \lor \neg \varphi}_v = 1 \]
\[ \Leftrightarrow \thickspace \val{\varphi}_v = 1 \thickspace or \thickspace \val{\neg \varphi}_v = 1 \]
\[ \Leftrightarrow \thickspace \val{\varphi}_v = 1 \thickspace or \thickspace \val{\varphi}_v = 0 \] \qed
\item $\vDash \neg\neg \varphi \to \varphi$
\[ \forall v \thickspace \val{\neg\neg \varphi \to \varphi}_v = 1 \]
\[ \Iff \thickspace \val{\neg\neg\varphi}_v = 0 \thickspace or \thickspace \val{\varphi}_v = 1 \]
\[ \Iff \thickspace \val{\neg \varphi}_v = 1 \thickspace or \thickspace \val{\varphi}_v = 1 \]
\[ \Iff \thickspace \val{\varphi}_v = 0 \thickspace or \thickspace \val{\varphi}_v = 1 \] \qed
\item $\vDash (\varphi \to \psi) \lor (\psi \to \varphi)$
\[ \forall v \thickspace \val{(\varphi \to \psi) \lor (\psi \to \varphi)}_v = 1 \]
\[ \Iff \val{\varphi \to \psi}_v = 1 \thickspace or \thickspace \val{\psi \to \varphi}_v = 1 \]
\[ \Iff (\val{\varphi}_v = 0 \thickspace or \thickspace \val{\psi}_v = 1) \thickspace oppure \thickspace (\val{\psi}_v = 0 \thickspace or \thickspace \val{\varphi}_v = 1) \]
\qed
\end{itemize}

\section{Soddisfacibilit\`{a}}
$v \vDash \varphi$ si legge `$v$ soddisfa $\varphi$, oppure `da $v$ segue $\varphi$, e significa che $\val{\varphi}_v = 1 $, ovvero
\[ \exists v \thinspace\text{ per cui la formula $\varphi$ \`{e} vera.} \]
\subsection{Esempi di soddisfacibilit\`{a}}
\begin{itemize}
\item $v \vDash p_0 \Iff v(p_0)=1$
\item $v \vDash p_0 \wedge p_1 \Iff v(p_0) = 1 and v(p_1) = 1$
\end{itemize}

\section{Teorema di correttezza e completezza}
Il teorema dice che se si dimostra semanticamente una conseguenza logica, allora \`{e} possibile trovarne una derivazione nella deduzione naturale.
\[ \vDash \leftrightarrow \vdash \]
Ovvero mette in corrispondenza \emph{sintassi} e \emph{semantica}.
\newpage
\section{Lemma}
\begin{enumerate}
\item $\Gamma \vDash \psi$\footnote[1]{Ipotesi A} e $\Delta, \psi$\footnote[2]{Ipotesi B}$ \vDash \varphi \Rightarrow \Gamma, \Delta \vDash \varphi$
\subsubsection{Dimostrazione}
\[\text{Per ogni valutazione $v$, se } \val{\Gamma}_v = \val{\Delta}_v = 1, \thickspace \text{allora } \val{\varphi}_v = 1\] Infatti:
\begin{equation}
\Gamma, \Delta \vDash \varphi
\begin{cases}
\text{Se } \Gamma \vDash \psi & allora \thickspace \Gamma, \Delta \vDash \psi \\ 
\text{Se } \Delta, \psi \vDash \varphi 
\end{cases}
\end{equation}
\[ \val{\Gamma}_v = \val{\Delta}_v = 1 \Rightarrow\text{\footnote[1]{}} \val{\psi}_v = 1 \Rightarrow\text{\footnote[2]{}} \val{\varphi}_v = 1 \]
\qed
\item $\Gamma \vDash \varphi \lor \psi$ e $\Delta, \psi \vDash \gamma$ e $\Sigma, \psi \vDash \gamma \Rightarrow \Gamma, \Delta, \Sigma \vDash \gamma$
\subsubsection{Dimostrazione} 
\[\text{Per ogni $v$, se } \val{\Gamma}_v=\val{\Delta}_v=\val{\Sigma}_v = 1\]
\[\text{allora } \val{\gamma}_v = 1 \]
\[\text{Se }\thickspace \val{\Gamma}_v = 1 \Rightarrow \val{\varphi \lor \psi}_v=1 \Iff \val{\varphi}_v = 1 \thinspace \text{ or } \thinspace \val{\psi}_v = 1 \]
\[\val{\varphi}_v = 1 \thinspace \text{ e } \thinspace \val{\Delta} = 1 \Rightarrow \val{\gamma}_v = 1 \]
oppure
\[\val{\psi}_v = 1 \thinspace \text{ e } \thinspace \val{\Sigma}_v = 1 \Rightarrow \val{\gamma}_v = 1\]
\[ \Iff \forall v \text{ t. c. } \val{\Delta}_v = 1 \thinspace \text{ e } \thinspace \val{\Sigma}_v = 1 \thinspace \text{ e } \thinspace \val{\varphi \lor \psi}_v = 1 \Rightarrow \val{\gamma}_v = 1 \]
\[ \Iff \Delta, \Sigma, \varphi \lor \psi \vDash \gamma \]
\[ \Iff\footnote[3]{Per Lemma punto 1. } \Delta, \Sigma, \Gamma \vDash \gamma \]
\qed
\end{enumerate}

\section{Relazione di equivalenza}

Definiamo una relazione di equivalenza con la notazione $\varphi \thickapprox \psi \Iff \val{\varphi}_v = \val{\psi}_v$.

\section{Sostituzione}

La sostituzione si denota con $\varphi[\psi / p_i]$ e significa che all'interno della formula $\varphi$ tutte le volte che compare $p_i$ va sostituito con $\psi$.
\subsubsection{Esempio:}
$\varphi \equiv p_1 \wedge p_0 \to (p_0 \to p_3)$, se scriviamo $\varphi[\neg p_0 \to p_3/ p_0]$, vogliamo sostituire ogni occorrenza di $p_0$ nella formula $\varphi$ con la formula $(\neg p_0 \to p_3)$, ovvero:
\[ \varphi[\neg p_0 \to p_3/ p_0] = p_1 \wedge (\neg p_0 \to p_3) \to ((\neg p_0 \to p_3) \to p_3) \]


% ============================ LEZIONE 6 - 19/10/23 ============================

\section {Definizione algebrica della semantica}

\begin{itemize}
\item $\val{\varphi \wedge \psi}_v = \val{\varphi}_v \cdot \val{\psi}_v$
\item $\val{\varphi \lor \psi}_v = \val{\varphi}_v + \val{\psi}_v$
\item $\val{\varphi \to \psi}_v = 1-\val{\varphi} + \val{\varphi}_v \cdot \val{\psi}$
\item $\val{\neg\varphi}_v = 1 - \val{\varphi}_v$
\item $\val{\varphi \leftrightarrow \psi}_v = 1 - |\val{\varphi}_v - \val{\psi}_v |$
\end{itemize}

\section{Propriet\`{a} della Semantica}

\subsection{Associativa}
La semantica gode della propriet\`{a} associativa:
\[(\varphi \wedge \psi) \wedge \sigma \leftrightarrow \varphi \wedge (\psi \wedge \sigma) \]
\[(\varphi \lor \psi) \lor \sigma \leftrightarrow \varphi \lor (\psi \lor \sigma) \]

\subsubsection{Dimostrazione $\vDash(\varphi \wedge \psi) \wedge \sigma \leftrightarrow \varphi \wedge (\psi \wedge \sigma)$}
\[ \forall v \thickspace \val{(\varphi \wedge \psi) \wedge \sigma}_v = \val{\varphi \wedge (\psi \wedge \sigma)}_v\]
\[\iff \val{\varphi \wedge \psi}_v = 1 \text{ and } \val{\sigma}_v=1\]
\[ \iff \val{\varphi}_v = 1 \text{ and } \val{\psi}_v = 1 \text{ and } \val{\sigma}_v = 1 \]

\subsection{Commutativa}
La semantica gode della propriet\`{a} commutativa:
\[ \varphi \wedge \psi \leftrightarrow \psi \wedge \varphi \]
\[ \varphi \lor \psi \leftrightarrow \psi \lor \varphi \]

\subsection{Distributiva}
La semantica gode della propriet\`{a} distributiva:
\[ \varphi \wedge (\psi \lor \sigma) \leftrightarrow (\varphi \wedge \psi) \lor (\varphi \wedge \sigma) \]
\[ \varphi \lor (\psi \wedge \sigma) \leftrightarrow (\varphi \lor \psi) \wedge (\varphi \lor \sigma) \]

\subsection{Legge di De Morgan}
Valgono le seguenti leggi:
\[ \neg(\varphi \wedge \psi) \leftrightarrow \neg \varphi \lor \neg \psi \]
\[ \neg (\varphi \lor \psi) \leftrightarrow \neg \varphi \wedge \neg \psi \]

\subsection {Idempotenza}
\[ \varphi \wedge \varphi \leftrightarrow \varphi \]
\[ \varphi \lor \varphi \leftrightarrow \varphi \]

\subsection{Legge della doppia negazione}
\[ \neg\neg\varphi \leftrightarrow \varphi \]

\subsection{Top, la formula sempre vera ($\top$)}
Si definisce la formula sempre vera come $\top \equiv \bot \to \bot$. In generale, si pu\`{o} dimostrare che $\vDash \bot \to \varphi \thickspace \forall \varphi \in PROP$. Infatti:
\begin{center}
\begin{tabular}{c c| c}
$\bot$ & $\varphi$ & $\bot \to \varphi$ \\
\hline
$0$ & $0$ & 1 \\
$0$ & $1$ & 1 
\end{tabular}
\end{center}
che semanticamente lo possiamo scrivere come segue:
\[ \val{\bot \to \varphi}_v = 1 \iff \val{\bot}_v = 0 \thickspace \text{ or }\thickspace \val{\varphi}_v = 1\]

% ============================ LEZIONE 7 - 23/10/23 ============================ 

\chapter{Deduzione Naturale}

\section{Concetto di Derivazione}

Nella Semantica abbiamo definito $\vDash$ come \emph{conseguenze logiche} e, nel caso in cui l'insieme delle premesse sia vuoto, come \emph{tautologie}. Nella \emph{deduzione naturale} possiamo definire delle \emph{derivazioni} da un insieme di ipotesi, oppure dei \emph{teoremi}, se l'insieme delle ipotesi \`{e} vuoto.
\[ \Gamma \vdash \varphi \Rightarrow \text{ significa che posso trovare una derivazione dalle formule in $\Gamma$ a $\varphi$}.\]
\[ \vdash \varphi \Rightarrow \text{ significa che $\varphi$ \`{e} un teorema}.\]

\`{E} possibile dimostrare che tautologie e teoremi coincidono.

\section{Le regole di derivazione}
\begin{center}
\begin{tikzpicture}
\draw (0,0) -- (2,0);
\node at (0,0) [above right]{$\varphi$};
\node at (2,0) [above left]{$\psi$};
\node at (1,0) [below] {$\sigma$};

\draw[->] (3,0.25) -- (2,0.25);
\node at(3,0) [above right]{premesse e ipotesi};

\draw[->] (3,-0.25) -- (2, -0.25);
\node at (3,0) [below right]{conclusioni};

\draw[->] (-2, 0) -- (-0.5, 0);
\node at(-2, 0.1) [left] {applicazione};
\node at(-2,-0.1)[below left] {di una regola};
\end{tikzpicture}
\end{center}

Questo \`{e} un esempio di derivazione, ma per poter dare un senso a questa notazione sintattica si devono introdurre delle regole di eliminazione ed introduzione dei connettivi.

\subsection{Regole di Eliminazione ed Introduzione}



\begin{multicols}{3}
\subsubsection{$\wedge I$ - AND Introduction}
\begin{center}
\begin{tikzpicture}
\draw (0,0) -- (2,0);
\node at (0,0) [above right] {$\varphi$};
\node at (2,0)[above left] {$\psi$};
\node at (1,0)[below] {$\varphi \wedge \psi$};
\end{tikzpicture}
\end{center}
\subsubsection{$\wedge E_1$ - AND Elimination}
\begin{center}
\begin{tikzpicture}
\draw (0,0) -- (2,0);
\node at (1,0)[above] {$\varphi \wedge \psi$};
\node at (1,0) [below] {$\varphi$};
\end{tikzpicture}
\end{center}
\subsubsection{$\wedge E_2$ - AND Elimination}
\begin{center}
\begin{tikzpicture}
\draw (0,0) -- (2,0);
\node at (1,0)[above] {$\varphi \wedge \psi$};
\node at (1,0)[below] {$\psi$};
\end{tikzpicture}
\end{center}
\end{multicols}
































\end{document}
